\section{Flow Around a Rotating Cylinder using OpenFOAM}

\subsection{Introduction}

This repository contains the setup files for numerical simulation project for studying the flow around a rotating cylinder using OpenFOAM. The two approaches that will be used to simulate the flow around the rotating cylinder are: 

\begin{itemize}
\item rotatingWallVelocity Function (Rotating Boundary Conditions {RBC})
\item Multiple Reference Frame (MRF)
\end{itemize}

OpenFOAM is an open-source CFD (Computational Fluid Dynamics) software designed to model fluid flows and other continuum phenomena. The simulation is set up to capture the details of the flow patterns around a cylinder rotating at a constant angular velocity.

\subsection{Requirements}

\begin{itemize}
\item OpenFOAM 
\end{itemize}

\subsection{Directory Structure}

!\href{Images/Directory.png}{An illustration of the OpenFOAM case setup, detailing the organization of directories and files required for running a simulation.}

The files depicted with blue icons - imply functional files within OpenFOAM which are used to perform a particular task. For instance, the MRFProperties files is used to setup the axis, region and speed of rotating region.

\subsection{Geometry}


![An illustration of the geometry designed from the works of Mittal and Kumar (2003)[1]](Images/Geo\_BC\_WhiteBackground.png)



\subsubsection{Mesh Generation}

\begin{itemize}
\item Meshing is done using Pointwise for the setup of this repository, but any meshing tool can be used since this is a 2D cylinder setup. 
\end{itemize}

\subsubsection{Boundary Conditions}

\begin{itemize}
\item In the \texttt{0/} directory, boundary conditions for \texttt{U}, \texttt{p}, and other necessary fields are specified.
\item Cylinder wall is set to rotate at a specified angular velocity.
\end{itemize}

\subsubsection{Solver Selection}

\begin{itemize}
\item The simulation is set up to use \texttt{pimpleFoam} for incompressible, transient flow.
\end{itemize}

\subsubsection{Control Parameters}

\begin{itemize}
\item Control parameters like time-step, write interval, and run-time are specified in \texttt{system/controlDict}.
\end{itemize}

\subsection{Running the Simulation}

1. Source OpenFOAM environment variables:

    \``\texttt{
    source /path/to/OpenFOAM/OpenFOAM-x.x/etc/bashrc
    \}`\texttt{

2. Navigate to the simulation root directory.

3. Generate the mesh:

    \}`\texttt{
    blockMesh
    \}`\texttt{

    (Optional: Refine mesh)

    \}`\texttt{
    snappyHexMesh -overwrite
    \}`\texttt{

4. Initialize the simulation:

    \}`\texttt{
    pimpleFoam
    \}`\texttt{

5. (Optional) Run the simulation using the shell script:

    \}`\texttt{
    chmod +x runSimulation.sh
    ./runSimulation.sh
    \}`\texttt{

\subsection{Post-Processing}

\begin{itemize}
\item Open the simulation results in ParaView to visualize the flow fields, vortex shedding, etc.
\item Optional post-processing scripts are located in the }postProcess/` directory.
\end{itemize}

\subsection{Contributing}

Feel free to fork the project and submit pull requests. Ensure you adhere to the coding and documentation guidelines.

\subsection{Authors}

\begin{itemize}
\item \href{mailto:youremail@example.com}{Your Name}
\end{itemize}

\subsection{Acknowledgements}

\begin{itemize}
\item OpenFOAM Foundation for the software.
\item \href{link}{Reference Paper}
\end{itemize}


\subsubsection{References}
\begin{itemize}
\item [1] Mittal, S., \& Kumar, B. (2003). *Flow past a rotating cylinder*. Journal of Fluid Mechanics, Volume(476), 303 - 334, url = https://api.semanticscholar.org/CorpusID:53349365.
\end{itemize}

\subsection{License}

This project is licensed under the MIT License. See the LICENSE.md file for details.

---

For more details, feel free to contact the author or refer to the project documentation.



!\href{./images/image\_name.jpg}{Alt text for image}

